\documentclass{article}
    % General document formatting
    \usepackage[margin=0.7in]{geometry}
    \usepackage[parfill]{parskip}
    \usepackage[utf8]{inputenc}
    
    % Related to math
    \usepackage{amsmath,amssymb,amsfonts,amsthm}
    \usepackage{multicol}
    \usepackage{commath}

    % Metadata for front page
    \title{Statistical Inference Study Notes}
    \author{Jakob Gerhard Martinussen}
    \date{2019, Spring}

    % Numbered environments
    \newtheorem{innertheorem}{Theorem}
    \newenvironment{theorem}[1]
      {\renewcommand\theinnertheorem{#1}\innertheorem}
      {\endinnertheorem}

    \newtheorem{innerdefinition}{Definition}
    \newenvironment{definition}[1]
      {\renewcommand\theinnerdefinition{#1}\innerdefinition}
      {\endinnerdefinition}

    % Aliases
    \renewcommand{\vec}[1]{\boldsymbol{#1}}
    \newcommand{\E}[1]{\mathrm{\mathb{E} \left[#1 \right]}}
    \newcommand{\Var}[1]{\mathrm{Var \left(#1 \right)}}

\begin{document}

\maketitle

\begin{multicols}{2}

\section{Probability Theory}

\section{Transformations and Expectations}

\begin{theorem}{2.3.15}[Moment generating functions for linear transformations]
  For any constants $a$ and $b$, the mgf of the random variable $aX + b$ is given by
  $$
  M_{aX + b}(t) = e^{bt} M_X(at)
  $$
\end{theorem}

\section{Common Families of Distributions}

\begin{theorem}{3.2.2}[Binomial Theorem]
  For any real numbers $x$ and $y$ and integer $n \geq 0$,
  $$
    (x + y)^n = \sum_{i = 0}^{n} \binom{n}{i} x^i y^{n-1}
  $$
\end{theorem}

\begin{definition}{3.4.1}[Exponential Family of Distributions]
  The exponential family consists of all pdfs and pmfs that can be expressed as
  $$
  f(x | \vec{\theta})
  =
  h(x) c(\vec{\theta}) \exp\left(
    \sum_{i = 1}^k w_i(\vec{\theta}) t_i(x)
  \right),
  $$
  where $h(x) \geq 0$ and $c(\vec{\theta}) \geq 0$.

  An alternative natural parametrization is
  $$
    f(x | \eta) = h(x) c^*(\eta) \exp \left( \sum_{i=1}^k \eta_i t_i(x) \right).
  $$
  The natural parameter space for the family is then given as
  $$
    \mathcal{H} = \left\{
      \eta = (\eta_1, ..., \eta_k):
        \int_{-\infty}^{\infty} h(x) \exp \left(
          \sum_{i=1}^k \eta_i t_i(x)
        \right)
    \right\},
  $$
  for which me must have
  $$
    c^*(\eta) = \left[
      \int_{-\infty}^{\infty}
        h(x) \exp \left( \sum_{i=1}^k \eta_i t_i(x) \right)
        \dif x
    \right]^{-1}.
  $$
\end{definition}

\begin{theorem}{3.4.2}[Exponential Family Statistics]
  \begin{align}
    \E{
      \sum_{i = 1}^k \pd{w_i(\vec{\theta})}{\theta_j} t_i(X)
    }
    &=
    -\pd{}{\theta_j} \log c(\vec{\theta})
    \\
    \Var{
      \sum_{i = 1}^k \pd{w_i(\vec{\theta})}{\theta_j} t_i(X)
    }
    &=
    -\pd[2]{}{\theta_j} \log c(\vec{\theta})
    -\E{\sum_{i = 1}^k \pd[2]{w_i(\vec{\theta})}{\theta_j} t_i(X)}
  \end{align}
\end{theorem}

\begin{definition}{3.4.7}[Curved Exponential Family]
  A curved exponential family is a family of densities of the form (3.4.1)
  for which the dimension of the vector $\vec{\theta}$ is equal to $d < k$.
  If $d = k$, the family is a full exponential family.
\end{definition}

\begin{theorem}{3.5.1}
  Let $f(x)$ be any pdf and let $\mu$ and $\sigma > 0$ be any given constants. Then the function
  $$
    g(x | \mu, \sigma) = \frac{1}{\sigma} f\del{\frac{x - \mu}{\sigma}}
  $$
  is a pdf.
\end{theorem}

\begin{definition}{3.5.2}[Location Family of Distributions]
  Let $f(x)$ be any pdf. Then the family of pdfs $f(x - \mu)$,
  indexed by the parameter $\mu$, $-\infty < \mu < \infty$,
  is called the location family with standard pdf $f(x)$
  and $-\mu$ is called the location parameter of the family.
\end{definition}

\begin{definition}{3.5.4}[Scale Family of Distributions]
  Let $f(x)$ be any pdf.
  Then for any $\sigma > 0$, the family of pdfs $\frac{1}{\sigma} f (\frac{x}{\sigma})$,
  indexed by the parameter $\sigma$, is called the scale family with standard pdf $f(x)$
  and $\sigma$ is called the scale parameter of the family.
\end{definition}

\begin{definition}{3.5.4}[Location-Scale Family of Distributions]
  Let $f(x)$ be any pdf.
  Then for any $\mu$, $-\infty < \mu < \infty$, and any $\sigma > 0$,
  the family of pdfs $\frac{1}{\sigma} f(\frac{x - \mu}{\sigma})$,
  indexed by the parameter $(\mu, \sigma)$,
  is called the location-scale family with standard pdf $f(x)$;
  $\mu$ is called the location parameter
  and $\sigma$ is called the scale parameter of the family.
\end{definition}

\begin{definition}{3.5.4}
  Let $f(\cdot)$ be any pdf. Let $\mu$ be any real number,
  and let $\sigma$ be any positive real number.
  Then $X$ is a random variable with pdf $\frac{1}{\sigma} f\del{\frac{x - \mu}{\sigma}}$
  if and only if there exists a random variable $Z$ with pdf $f(z)$ and $X = \sigma Z + \mu$.
\end{definition}

\begin{definition}{3.5.7}
  Let $Z$ be a random variable with pdf $f(z)$. Suppose $\E{Z}$ and $\Var{Z}$ exist.
  If $X$ is a random variable with pdf $\frac{1}{\sigma} f\del{\frac{x - \mu}{\sigma}}$,
  then
  $$
    \E{X} = \sigma \E{Z} + \mu,\text{ and } \Var{X} = \sigma^2 \Var{Z}.
  $$
  In particular, if $\E{Z} = 0$ and $\Var{Z} = 1$, then $\E{X} = \mu$ and $\Var{X} = \sigma^2$.
\end{definition}

\begin{theorem}{3.6.1}[Chebychev's Inequality]
  Let $X$ be a random variable and let $g(x)$ be a nonnegative function. Then, for any $r > 0$,
  $$
  P\del{g(X) \geq r} \geq \frac{\E{g(X)}}{r}.
  $$

  This can be used in order to find the following bounds
  \begin{align}
    P\del{\left| X - \mu \right| \geq t\sigma} &\leq \frac{1}{t^2}
    \\
    P\del{\left| X - \mu \right| < t\sigma} &\geq 1 - \frac{1}{t^2}
  \end{align}
\end{theorem}

\newpage
\appendix

\section{Appendix}

\subsubsection*{Binomial Coefficient Identities}

\begin{align}
  \binom{n}{x} &= \frac{n!}{x!(n-x)!}
  \\
  x \binom{n}{x} &= n \binom{n - 1}{x - 1}
  \\
  \binom{n}{x} &= \frac{n}{x} \binom{n - 1}{x - 1}
\end{align}

\subsubsection*{Gamma function}

\begin{align}
  \Gamma(\alpha + 1) &= \alpha \Gamma(\alpha) &\alpha > 0
  \\
  \Gamma(n) &= (n - 1)! &\n \text{positive integer}
  \\
  \Gamma \left( \frac{1}{2} \right) &= \sqrt{\pi}
\end{align}

\end{multicol}
\end{document}
