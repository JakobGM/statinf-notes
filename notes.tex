\documentclass{article}
    % General document formatting
    \usepackage[margin=0.7in]{geometry}
    \usepackage[parfill]{parskip}
    \usepackage[utf8]{inputenc}
    
    % Related to math
    \usepackage{amsmath,amssymb,amsfonts,amsthm}
    \usepackage{multicol}
    \usepackage{commath}

    % Better enumeration support
    \usepackage{enumitem}

    % Metadata for front page
    \title{Statistical Inference Study Notes}
    \author{Jakob Gerhard Martinussen}
    \date{2019, Spring}

    % Numbered environments
    \theoremstyle{plain}
    \newtheorem{innertheorem}{Theorem}
    \newenvironment{theorem}[1]
      {\renewcommand\theinnertheorem{#1}\innertheorem}
      {\endinnertheorem}

    \theoremstyle{definition}
    \newtheorem{innerdefinition}{Definition}
    \newenvironment{definition}[1]
      {\renewcommand\theinnerdefinition{#1}\innerdefinition}
      {\endinnerdefinition}

    \theoremstyle{plain}
    \newtheorem{innerlemma}{Lemma}
    \newenvironment{lemma}[1]
      {\renewcommand\theinnerlemma{#1}\innerlemma}
      {\endinnerlemma}

    \theoremstyle{plain}
    \newtheorem{innercorollary}{Corollary}
    \newenvironment{corollary}[1]
      {\renewcommand\theinnercorollary{#1}\innercorollary}
      {\endinnercorollary}

    % Aliases
    \renewcommand{\vec}[1]{\boldsymbol{#1}}
    \newcommand{\E}[1]{\mathrm{\mathbb{E} \left[#1 \right]}}
    \newcommand{\Var}[1]{\mathrm{Var \left(#1 \right)}}

\begin{document}

\maketitle

\begin{multicols}{2}

\section{Probability Theory}

\begin{definition}{1.3.2}[Conditional Probability]
  If $A$ and $B$ are events in $S$, and $P(B) > 0$,
  then the \textit{conditional probability of $A$ given $B$},
  written $P(A | B)$ is
  \begin{equation*}
    P(A|B) = \frac{P(A \cap B)}{P(B)}.
  \end{equation*}
\end{definition}

\section{Transformations and Expectations}

\begin{theorem}{2.1.5}[] 
  Let $X$ have pdf $f_X(x)$ and let $Y = g(X)$, where $g$ is a monotone function.
  Let $\mathcal{X}$ and $\mathcal{Y}$ be defined by
  \begin{align*}
    \mathcal{X} &= \{x: f_X(x) > 0\}
    \\
    \mathcal{Y} &= \{ y: y = g(x) \text{ for some } x \in \mathcal{X} \}
  \end{align*}
  Suppose that $f_X(x)$ is continous on $\mathcal{X}$ and that $g^{-1}(x)$
  has a continous derivative and $\mathcal{Y}$. Then the pdf $Y$ is given by
  \[
    f_Y(y) = \begin{cases}
      f_X(g^{-1}(y)) \left| \od{}{y} g^{-1}(y) \right|, &y \in \mathcal{Y}
      \\
      0, &\text{otherwise}.
    \end{cases}
  \]
\end{theorem}

\begin{theorem}{2.3.15}[Moment generating functions for linear transformations]
  For any constants $a$ and $b$, the mgf of the random variable $aX + b$ is given by
  $$
  M_{aX + b}(t) = e^{bt} M_X(at)
  $$
\end{theorem}

\section{Common Families of Distributions}

\begin{theorem}{3.2.2}[Binomial Theorem]
  For any real numbers $x$ and $y$ and integer $n \geq 0$,
  $$
    (x + y)^n = \sum_{i = 0}^{n} \binom{n}{i} x^i y^{n-1}
  $$
\end{theorem}

\begin{definition}{3.4.1}[Exponential Family of Distributions]
  The exponential family consists of all pdfs and pmfs that can be expressed as
  $$
  f(x | \vec{\theta})
  =
  h(x) c(\vec{\theta}) \exp\left(
    \sum_{i = 1}^k w_i(\vec{\theta}) t_i(x)
  \right),
  $$
  where $h(x) \geq 0$ and $c(\vec{\theta}) \geq 0$.

  An alternative natural parametrization is
  $$
    f(x | \eta) = h(x) c^*(\eta) \exp \left( \sum_{i=1}^k \eta_i t_i(x) \right).
  $$
  The natural parameter space for the family is then given as
  $$
    \mathcal{H} = \left\{
      \eta = (\eta_1, ..., \eta_k):
        \int_{-\infty}^{\infty} h(x) \exp \left(
          \sum_{i=1}^k \eta_i t_i(x)
        \right)
    \right\},
  $$
  for which me must have
  $$
    c^*(\eta) = \left[
      \int_{-\infty}^{\infty}
        h(x) \exp \left( \sum_{i=1}^k \eta_i t_i(x) \right)
        \dif x
    \right]^{-1}.
  $$
\end{definition}

\begin{theorem}{3.4.2}[Exponential Family Statistics]
  \begin{align}
    \E{
      \sum_{i = 1}^k \pd{w_i(\vec{\theta})}{\theta_j} t_i(X)
    }
    &=
    -\pd{}{\theta_j} \log c(\vec{\theta})
    \\
    \Var{
      \sum_{i = 1}^k \pd{w_i(\vec{\theta})}{\theta_j} t_i(X)
    }
    =
    &-\pd[2]{}{\theta_j} \log c(\vec{\theta})
    \\
    &-\E{\sum_{i = 1}^k \pd[2]{w_i(\vec{\theta})}{\theta_j} t_i(X)}
  \end{align}
\end{theorem}

\begin{definition}{3.4.7}[Curved Exponential Family]
  A curved exponential family is a family of densities of the form (3.4.1)
  for which the dimension of the vector $\vec{\theta}$ is equal to $d < k$.
  If $d = k$, the family is a full exponential family.
\end{definition}

\begin{theorem}{3.5.1}
  Let $f(x)$ be any pdf and let $\mu$ and $\sigma > 0$ be any given constants. Then the function
  $$
    g(x | \mu, \sigma) = \frac{1}{\sigma} f\del{\frac{x - \mu}{\sigma}}
  $$
  is a pdf.
\end{theorem}

\begin{definition}{3.5.2}[Location Family of Distributions]
  Let $f(x)$ be any pdf. Then the family of pdfs $f(x - \mu)$,
  indexed by the parameter $\mu$, $-\infty < \mu < \infty$,
  is called the location family with standard pdf $f(x)$
  and $-\mu$ is called the location parameter of the family.
\end{definition}

\begin{definition}{3.5.4}[Scale Family of Distributions]
  Let $f(x)$ be any pdf.
  Then for any $\sigma > 0$, the family of pdfs $\frac{1}{\sigma} f (\frac{x}{\sigma})$,
  indexed by the parameter $\sigma$, is called the scale family with standard pdf $f(x)$
  and $\sigma$ is called the scale parameter of the family.
\end{definition}

\begin{definition}{3.5.4}[Location-Scale Family of Distributions]
  Let $f(x)$ be any pdf.
  Then for any $\mu$, $-\infty < \mu < \infty$, and any $\sigma > 0$,
  the family of pdfs $\frac{1}{\sigma} f(\frac{x - \mu}{\sigma})$,
  indexed by the parameter $(\mu, \sigma)$,
  is called the location-scale family with standard pdf $f(x)$;
  $\mu$ is called the location parameter
  and $\sigma$ is called the scale parameter of the family.
\end{definition}

\begin{definition}{3.5.4}
  Let $f(\cdot)$ be any pdf. Let $\mu$ be any real number,
  and let $\sigma$ be any positive real number.
  Then $X$ is a random variable with pdf $\frac{1}{\sigma} f\del{\frac{x - \mu}{\sigma}}$
  if and only if there exists a random variable $Z$ with pdf $f(z)$ and $X = \sigma Z + \mu$.
\end{definition}

\begin{definition}{3.5.7}
  Let $Z$ be a random variable with pdf $f(z)$. Suppose $\E{Z}$ and $\Var{Z}$ exist.
  If $X$ is a random variable with pdf $\frac{1}{\sigma} f\del{\frac{x - \mu}{\sigma}}$,
  then
  $$
    \E{X} = \sigma \E{Z} + \mu,\text{ and } \Var{X} = \sigma^2 \Var{Z}.
  $$
  In particular, if $\E{Z} = 0$ and $\Var{Z} = 1$, then $\E{X} = \mu$ and $\Var{X} = \sigma^2$.
\end{definition}

\begin{theorem}{3.6.1}[Chebychev's Inequality]
  Let $X$ be a random variable and let $g(x)$ be a nonnegative function. Then, for any $r > 0$,
  $$
  P\del{g(X) \geq r} \geq \frac{\E{g(X)}}{r}.
  $$

  This can be used in order to find the following bounds
  \begin{align}
    P\del{\left| X - \mu \right| \geq t\sigma} &\leq \frac{1}{t^2}
    \\
    P\del{\left| X - \mu \right| < t\sigma} &\geq 1 - \frac{1}{t^2}
  \end{align}
\end{theorem}

\section{Multiple Random Variables}

\begin{definition}{4.1.1}[n-dimensional random vector]
  An \textit{n-dimensional random vector} is a function from a sample space $S$ into $\mathfrak{R}^n$,
  $n$-dimensional Euclidean space.
\end{definition}

\begin{definition}{4.1.3}[Joint probability mass function]
  Let $(X, Y)$ be a discrete bivariate random vector.
  Then the function $f(x, y)$ from $\mathfrak{R}^2$ into $\mathfrak{R}$ defined by
  $f(x, y) = P(X = x, Y = y)$ is called the \textit{joint probability mass function} or
  \textit{joint pmf} of $(X, Y)$. If it is necessary to stress the fact that $f$ is the
  joint pmf of the vector $(X, Y)$ rather than som other vector, the notation $f_{X,Y}(x, y)$
  will be used.
\end{definition}

\begin{theorem}{4.1.6}[Marginal distribution]
  Let $(X, Y)$ be a discrete bivariate random vector with joint pmf $f_{X,Y}(x,y)$.
  Then the marginal pmfs of $X$ and $Y$, $f_{X}(x) = P(X = x)$ and $f_{Y}(y) = P(Y = y)$,
  are given by
  \begin{equation*}
    f_{X}(x) = \sum_{y \in \mathfrak{R}} f_{X,Y}(x,y) 
    ~~~\text{ and }~~~
    f_{Y}(y) = \sum_{x \in \mathfrak{R}} f_{X,Y}(x,y) 
  \end{equation*}
\end{theorem}

\begin{definition}{4.1.10}[Joint probability density function]
  A function $f(x, y)$ from $\mathfrak{R}^2$ into $\mathfrak{R}$ is called a
  \textit{joint probability density function} or \textit{joint pdf of the continuous bivariate
  random vector} $(X, Y)$ if, for every $A \subset \mathfrak{R}^2$,
  \begin{equation*}
    P((X,Y) \in A) = \int \limits_{A} \int f(x,y) \dif{x} \dif{y}
  \end{equation*}
\end{definition}

\begin{definition}{4.2.1}[Conditional PMF]
  Let $(X,Y)$ be a discrete bivariate random vector with joint pmf $f(x,y)$ and marginal pmfs
  $f_{X}(x)$ and $f_{Y}(y)$. For any $x$ such that $P(X=x) = f_{X}(x) > 0$,
  the \textit{conditional pmf of $Y$ given that $X = x$} is the function of $y$
  denoted by $f(y | x)$ and defined by
  \begin{equation*}
    f(y|x) = P(Y = y | X = x) = \frac{f(x, y)}{f_{X}(x)}.
  \end{equation*}
  For any $y$ such that $P(Y=y) = f_{Y}(y) > 0$,
  the \textit{conditional pmf of $X$ given that $Y = y$} is the function of $x$
  denoted by $f(x | y)$ and defined by
  \begin{equation*}
    f(x|y) = P(X = x | Y = y) = \frac{f(x, y)}{f_{Y}(y)}.
  \end{equation*}
\end{definition}

\begin{definition}{4.2.1}[Conditional PDF]
  Let $(X,Y)$ be a continous bivariate random vector with joint pdf $f(x,y)$ and marginal pdfs
  $f_{X}(x)$ and $f_{Y}(y)$. For any $x$ such that $f_{X}(x) > 0$,
  the \textit{conditional pdf of $Y$ given that $X = x$} is the function of $y$
  denoted by $f(y | x)$ and defined by
  \begin{equation*}
    f(y|x) = \frac{f(x, y)}{f_{X}(x)}.
  \end{equation*}
  For any $y$ such that $f_{Y}(y) > 0$,
  the \textit{conditional pmf of $X$ given that $Y = y$} is the function of $x$
  denoted by $f(x | y)$ and defined by
  \begin{equation*}
    f(x|y) = \frac{f(x, y)}{f_{Y}(y)}.
  \end{equation*}
\end{definition}

\begin{definition}{4.2.5}[Independent Random Variables]
  Let $(X, Y)$ be a bivariate random vector with joint pdf or pmf $f(x, y)$
  and marginal pdfs or pmfs $f_X(x)$ and $f_Y(y)$.
  Then $X$ and $Y$ are called \textit{independent random variables} if,
  for every $x \in \mathfrak{R}$ and $y \in \mathfrak{R}$,
  \begin{equation}
    f(x, y) = f_X(x) f_Y(y).
  \end{equation}
\end{definition}

\begin{lemma}{4.2.7}[Verifying Independence]
  Let $(X,Y)$ be a bivariate  random vector with joint pdf or pmf $f(x,y)$.
  Then $X$ and $Y$ are independent random variables if and only if there exist functions
  $g(x)$ and $h(y)$ such that, for every $x \in \mathfrak{R}$ and $y \in \mathfrak{R}$,
  \begin{equation*}
    f(x, y) = g(x) h(y)
  \end{equation*}
\end{lemma}

\begin{theorem}{4.2.10}[Independent Events and Expectations]
  Let $X$ and $Y$ be independent random variables.
  \begin{enumerate}[label=\alph*.]
    \item For any $A \subset \mathfrak{R}$ and $B \subset \mathfrak{R}$,
      $P(X \in A, Y \in B) = P(X \in A) P(Y \in B)$, that is, the events
      $\{X \in A\}$ and $\{Y \in B\}$ are independent events.
    \item Let $g(x)$ be a function only of $x$ and $h(y)$ be a function only of $y$. Then
      \begin{equation*}
        \E{g(X) h(Y)} = \E{g(X)} \E{h(Y)}.
      \end{equation*}
  \end{enumerate}
\end{theorem}

\begin{theorem}{4.2.12}[MGF of Sum of Independent Variables]
  Let $X$ and $Y$ be independent random variables with moment generating functions
  $M_{X}(t)$ and $M_{Y}(t)$. Then the moment generating function of the random variable
  $Z = X + Y$ is given by
  \begin{equation*}
    M_Z(t) = M_X(t) M_Y(t).
  \end{equation*}
\end{theorem}

\begin{theorem}{4.2.14}[Sum of Two Random Normal Variables]
  Let $X \sim \mathcal{N}(\mu, \sigma^{2})$ and $Y \sim \mathcal{N}(\gamma, \tau^{2})$
  be independent normal random variables. Then the random variable $Z = X + Y$ has
  a $\mathcal{N}(\mu + \tau, \sigma^{2} + \tau^{2})$ distribution.
\end{theorem}

\begin{theorem}{4.3.2}[]
  If $X \sim \text{Poisson}(\theta)$ and $Y \sim \text{Poisson}(\lambda)$
  and $X$ and $Y$ are independent, then $X + Y \sim \text{Poisson}(\theta + \lambda)$.
\end{theorem}

\begin{theorem}{4.4.3}[Law of Double Expectation]
  If $X$ and $Y$ are any two random variables, then
  \begin{equation*}
    \E{X} = \E{\E{X | Y}},
  \end{equation*}
  provided that the expectations exist.
\end{theorem}

\begin{theorem}{4.4.7}[Conditional Variance Identity]
  For any two random variables $X$ and $Y$,
  \begin{equation*}
    \Var{X} = \E{\Var{X | Y}} + \Var{\E{X | Y}},
  \end{equation*}
  provided that the expectations exist.
\end{theorem}

\begin{definition}{4.5.1}[Covariance]
  The \textit{covariance of $X$ and $Y$} is the number defined by
  \begin{equation*}
    Cov(X, Y) = \E{(X - \mu_X)(Y - \mu_Y)}.
  \end{equation*}
\end{definition}

\begin{definition}{4.5.2}[Correlation]
  The \textit{correlation of $X$ and $Y$} is the number defined by
  \begin{equation*}
    \rho_{XY} = \frac{\text{Cov}(X, Y)}{\sigma_X \sigma_Y}.
  \end{equation*}
  The value $\rho_{XY}$ is also called the \textit{correlation coefficient}.
\end{definition}

\begin{theorem}{4.5.3}
  For any random variables $X$ and $Y$,
  \begin{equation*}
    \text{Cov}(X, Y) = \E{XY} - \mu_X \mu_Y
  \end{equation*}
\end{theorem}

\begin{theorem}{4.5.5}
  If $X$ and $Y$ are independent random variables, then $\text{Cov}(X, Y) = 0$
  and $\rho_{XY} = 0$.
\end{theorem}

\begin{theorem}{4.5.6}
  If $X$ and $Y$ are any two random variables and $a$ and $b$ are any two
  constants, then
  \begin{equation*}
    \text{Var}(aX + bY)
    =
    a^{2} \text{Var}(X) + b^{2} \text{Var}(Y) + 2ab \text{Cov}(X, Y)
  \end{equation*}
\end{theorem}

\begin{definition}{4.4.4}[Mixture Distribution]
  A random variable $X$ is said to have a \textit{mixture distribution}
  if the distribution of $X$ depends on a quantity that also has a distribution.
\end{definition}

\begin{theorem}{4.5.7}
  For any random variables $X$ and $Y$,
  \begin{enumerate}[label=\alph*.]
    \item $-1 \leq \rho_{XY} \leq 1$.
    \item $|\rho_{XY}| = 1$ if and only if there exist numbers $a \neq 0$
      and $b$ such that $P(Y = aX + b) = 1$.
      If $\rho_{XY} = 1$, then $a > 0$, and if $\rho_{XY} = -1$, then $a < 0$.
  \end{enumerate}
\end{theorem}

\begin{definition}{4.6.2}[Multinomial Distribution]
  Let $n$ and $m$ be positive integers and let $p_1, ..., p_n$ be numbers satisfying
  $0 \leq p_i \leq 1, i = 1, ..., n$ and $\sum_{i=1}^n p_i = 1$.
  Then the random vector $(X_1, ..., X_n)$ has a
  \textit{multinomial distribution with $m$ trials and cell probabilities $p_1, ..., p_n$}
  if the joint pmf of $(X_1, ..., X_n)$ is
  \begin{equation*}
    f(x_1, ..., x_n)
    = \frac{m!}{x_1!...x_n!} p_1^{x_1} ... p_n^{x_n}
    = m! \prod_{i=1}^n \frac{p_i^{x_i}}{x_i!}
  \end{equation*}
  on the set of $(x_1, ..., x_n)$ such that each $x_i$ is a nonnegative integer
  and $\sum_{i = 1}^n x_i = m$.
\end{definition}

\begin{theorem}{4.6.2}[Multinomial Theorem]
  Let $m$ and $n$ be positive integers.
  Let $\mathcal{A}$ be the set of vectors $\vec{x} = (x_1, ..., x_n)$
  such that each $x_i$ is a nonnegative integer
  and $\sum \limits_{i=1}^{n} x_i = m$. Then, for any real numbers $p_1, ..., p_n$,
  \begin{equation*}
    (p_1 + ... + p_n)^m
    =
    \sum_{x \in \mathcal{A}} \frac{m!}{x_1! ... x_n!} p_1^{x_1} ... p_n^{x_n}.
  \end{equation*}
\end{theorem}

\begin{definition}{4.6.5}
  Let $\vec{X}_1, ..., \vec{X}_n$ be random vectors with joint pdf or pmf
  $f(\vec{x}_1, ..., \vec{x}_n)$. Let $f_{\vec{X}_i}(\vec{x}_i)$ denote the
  marginal pdf or pmf of $\vec{X}_i$. Then $\vec{X_1}, ..., \vec{X_n}$
  are called \textit{mutually independent random vectors} if, for every
  $(\vec{x}_1, ..., \vec{x}_n)$,
  \begin{equation*}
    f(\vec{x}_1, ..., \vec{x}_n) = \prod_{i = 1}^{n} f_{\vec{X}_i}(\vec{x}_i).
  \end{equation*}
  If the $X_i$s are alle one-dimensional, then $X_1, ..., X_n$ are called
  \textit{mutually independent random variables}.
\end{definition}

\begin{theorem}{4.6.6}[]
  Let $X_1, ..., X_n$ be mutually independent random variables.
  Let $g_1, ..., g_n$ be real-valued functions such that $g_i(x_i)$ is a function
  only of $x_i, i = 1, ..., n$.
  Then
  \begin{equation*}
    \E{g_1(X_1) \cdot ... \cdot g_n(X_n)}
    =
  \E{g_1(X_1)} \cdot ... \cdot \E{g_n(X_n)}
  \end{equation*}
\end{theorem}

\begin{theorem}{4.6.7}[]
  Let $X_1, ..., X_n$ be mutually independent random variables with mgfs
  $M_{X_1}(t), ..., M_{X_n}(t)$. Let $Z = X_1 + ... + X_n$.
  Then the mgf of $Z$ is
  \begin{equation*}
    M_Z(t) = M_{X_1}(t) \cdot ... \cdot M_{X_n}(t).
  \end{equation*}
  In particular, if $X_1, ..., X_n$ all have the same distribution with mgf
  $M_X(t)$, then
  \begin{equation*}
    M_Z(t) = \del{M_X(t)}^n.
  \end{equation*}
\end{theorem}

\begin{corollary}{4.6.9}[]
  Let $X_1, ..., X_n$ be mutually independent random variables with mgfs
  $M_{X_1}(t), ..., M_{X_n}(t)$. Let $a_1, ..., a_n$ and $b_1, ..., b_n$
  bef fixed constants. Let $Z = (a_1 X_1 + b_1) + ... + (a_n X_n + b_n)$.
  Then the mgf of $Z$ is
  \begin{equation*}
    M_Z(t) = \del{e^{t\del{\sum b_i}}} M_{X_1}(a_1 t) \cdot ... \cdot M_{X_n}(a_n t)
  \end{equation*}
\end{corollary}

\begin{corollary}{4.6.10}[]
  Let $X_1, ..., X_n$ be mutually independent random variables with
  $X_i \sim \mathcal{N}(\mu_i, \sigma_i^{2})$. Let $a_1, ..., a_n$
  and $b_1, ..., b_n$ be fixed constants. Then
  \begin{equation*}
    Z = \sum_{i = 1}^{n} \del{a_i X_i + b_i}
    \sim
    \mathcal{N}\del{
      \sum_{i = 1}^{n} (a_i \mu_i + b_i), \sum_{i = 1}^{n} a_i^{2} \sigma_i^{2}
    }
  \end{equation*}
\end{corollary}

\begin{theorem}{4.6.11}[]
  Let $\vec{X}_1, ..., \vec{X}_n$ be random vectors.
  Then $\vec{X}_1, ..., \vec{X}_n$ are mutually independent vectors if and only if
  there exist functions $g_i(\vec{x}_i), i = 1, ..., n$, such that the joint pdf or pmf of
  $(\vec{X}_1, ..., \vec{X}_n)$ can be written as
  \begin{equation*}
    f(\vec{x}_1, ..., \vec{x}_n) = g_1(\vec{x}_1) \cdot ... \cdot g_n(\vec{x}_n).
  \end{equation*}
\end{theorem}

\begin{theorem}{4.6.12}[]
  Let $\vec{X}_1, ..., \vec{X}_n$ be independent random vectors. Let $g_i(\vec{x}_i)$
  be a function only of $\vec{x}_i, i = 1, ..., n$. Then the random variables
  $U_i = g_i(\vec{X}_i), i = 1, ..., n$, are mutually independent.
\end{theorem}

\begin{theorem}{4.6.13}[]
  Let $(X_1, ..., X_n)$ be a random vector with pdf $f(x_1, ..., x_n)$.
  Consider a new random vector $(U_1, ..., U_n)$, defined by
  $U_j = g_j(X_1, ..., X_n), j = 1, ..., n$.
  Denote the $i$th inverse by $x_1 = h_{1i}(u_1, ..., u_n)$.
  Then
  \begin{align*}
    & f_{\vec{U}}(u_1, ..., u_n)
    \\
    & ~~~~ = \sum_{i = 1}^{k} f_{\vec{X}}(h_{1i}(u_1, ..., u_n), ..., h_{ni}(u_1, ..., u_n)) |J_i|,
  \end{align*}
  where
  \begin{equation*}
    \del{J_i}_{jk} = \pd{x_i}{u_k} = \pd{h_{ji}}{u_k}.
  \end{equation*}
\end{theorem}

\begin{theorem}{4.7.3}[Cauchy-Schwarz Inequality]
  For any two random variables $X$ and $Y$,
  \begin{equation*}
    |\E{XY}| \leq \E{|XY|} \leq \sqrt{\E{|X|^2}} \sqrt{\E{|Y|^2}}
  \end{equation*}
\end{theorem}

\begin{theorem}{4.7.7}[Jensen's Inequality]
  For any random variable $X$, if $g(x)$ is a convex function, then
  \begin{equation*}
    \E{g(X)} \geq g\del{\E{X}}
  \end{equation*}
  Equality holds if and only if, for every line $a + bx$ that is tangent to $g(x)$ at $x = \E{X}$, $P(g(X) = a + bX) = 1$.
\end{theorem}

\newpage
\appendix

\section{Appendix}

\subsubsection*{Binomial Coefficient Identities}

\begin{align}
  \binom{n}{x} &= \frac{n!}{x!(n-x)!}
  \\
  x \binom{n}{x} &= n \binom{n - 1}{x - 1}
  \\
  \binom{n}{x} &= \frac{n}{x} \binom{n - 1}{x - 1}
\end{align}

\subsubsection*{Gamma function}

\begin{align}
  \Gamma(\alpha + 1) &= \alpha \Gamma(\alpha) &\alpha > 0
  \\
  \Gamma(n) &= (n - 1)! & n \text{ positive integer}
  \\
  \Gamma \left( \frac{1}{2} \right) &= \sqrt{\pi}
\end{align}

\subsubsection*{Distribution of the variance estimator}
Let $Z_1, Z_2, ..., Z_n$ be $\mathcal{N}(\mu, \sigma^{2}$
and let
\[
  S^2 = \frac{1}{n-1} \sum_{i = 1}^{n} (Z_i - \bar{Z})^{2},
\]
be the variance estimator.
Then
\[
  \frac{(n-1)S^2}{\sigma^{2}} \sim \chi^2_{n-1}.
\]

\end{multicols}
\end{document}
